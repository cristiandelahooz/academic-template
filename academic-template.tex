% =============================================================================
% TEMPLATE ACADÉMICO PARA TRABAJOS EDUCATIVOS
% =============================================================================
% Este template proporciona una estructura completa y profesional para trabajos
% académicos, ensayos, cuestionarios, reportes y otros documentos educativos.
% 
% INSTRUCCIONES DE USO:
% 1. Personalizar las variables de documento en la sección "CONFIGURACIÓN DEL DOCUMENTO"
% 2. Modificar colores en la sección "CONFIGURACIÓN DE COLORES" si es necesario
% 3. Agregar contenido en las secciones correspondientes
% 4. Ejecutar: make build (o pdflatex + biber + pdflatex + pdflatex)
%
% COMPATIBILIDAD LLM:
% - Todas las variables están claramente documentadas
% - Estructura modular para fácil comprensión
% - Comentarios explicativos en cada sección
% =============================================================================

\documentclass[12pt]{article}

% =============================================================================
% CONFIGURACIÓN DE PÁGINA Y GEOMETRÍA
% =============================================================================
% Establece el tamaño de página, márgenes y espaciado general del documento
\usepackage[letterpaper, margin=1in, headheight=0.25in, headsep=0.5in]{geometry}
\setlength{\footskip}{33.5pt} % Espacio para el pie de página

% =============================================================================
% CONFIGURACIÓN DE IDIOMA Y CODIFICACIÓN
% =============================================================================
% Configuración para documentos en español con codificación UTF-8
\usepackage[spanish,es-tabla,es-nodecimaldot]{babel} % Idioma español
\usepackage[utf8]{inputenc}     % Codificación de entrada UTF-8
\usepackage[T1]{fontenc}        % Codificación de fuentes T1
\usepackage{lmodern}            % Fuente Latin Modern mejorada

% =============================================================================
% PAQUETES PARA GRÁFICOS Y TABLAS
% =============================================================================
% Herramientas para inserción de imágenes y creación de tablas profesionales
\usepackage{graphicx}           % Inserción de imágenes
\usepackage{tcolorbox}          % Cajas de texto con color
\usepackage{booktabs}           % Tablas profesionales
\usepackage{tabularray}         % Tablas avanzadas
\usepackage{tabularx}           % Tablas con ancho automático

% =============================================================================
% CONFIGURACIÓN DE FORMATO Y ESPACIADO
% =============================================================================
% Controla el espaciado entre párrafos y otros elementos de formato
\usepackage{parskip}            % Espaciado automático entre párrafos
\usepackage{needspace}          % Control de saltos de página
\usepackage{calc}               % Cálculos matemáticos en LaTeX

% =============================================================================
% PAQUETES PARA COLORES Y DISEÑO VISUAL
% =============================================================================
% Herramientas para personalización visual del documento
\usepackage{xcolor}             % Soporte para colores avanzados
\usepackage{tikz}               % Gráficos vectoriales para headers/footers

% =============================================================================
% CONFIGURACIÓN DE ENCABEZADOS Y PIES DE PÁGINA
% =============================================================================
% Sistema de headers y footers personalizados
\usepackage{fancyhdr}           % Encabezados y pies de página personalizados
\usepackage{lastpage}           % Referencia a la última página

% =============================================================================
% CONFIGURACIÓN DE TÍTULOS Y SECCIONES
% =============================================================================
% Personalización del formato de títulos y secciones
\usepackage{titlesec}           % Personalización de títulos de sección

% =============================================================================
% CONFIGURACIÓN DE ÍNDICES Y TABLA DE CONTENIDOS
% =============================================================================
% Herramientas para generar índices profesionales
\usepackage{tocloft}            % Personalización de tabla de contenidos

% =============================================================================
% CONFIGURACIÓN MATEMÁTICA
% =============================================================================
% Soporte para fórmulas y expresiones matemáticas
\usepackage{amsmath}            % Entorno matemático avanzado

% =============================================================================
% CONFIGURACIÓN DE BIBLIOGRAFÍA
% =============================================================================
% Sistema de referencias bibliográficas con Biber/BibLaTeX
\usepackage[backend=biber, style=ieee, sortcites, url=true, autocite=superscript]{biblatex}
\usepackage{csquotes}           % Manejo de citas y comillas

% =============================================================================
% CONFIGURACIÓN DE HIPERVÍNCULOS
% =============================================================================
% Enlaces internos y externos en el documento
\usepackage{hyperref}           % Hipervínculos (debe cargarse al final)

% =============================================================================
% CONFIGURACIÓN DE COLORES DEL TEMPLATE
% =============================================================================
% Paleta de colores principal del documento. Modifica estos valores para
% personalizar la apariencia visual del template.

% Colores primarios (usados en títulos y elementos principales)
\definecolor{primaryColor}{RGB}{124, 10, 2}        % Rojo oscuro principal
\definecolor{secondaryColor}{RGB}{240, 89, 65}     % Rojo medio secundario  
\definecolor{accentColor}{RGB}{239, 167, 0}        % Naranja para acentos

% Colores funcionales
\definecolor{linkColor}{RGB}{239, 167, 0}          % Color para hipervínculos
\definecolor{headerTextColor}{RGB}{255, 255, 255}  % Texto en encabezados
\definecolor{bodyTextColor}{RGB}{0, 0, 0}          % Texto principal del documento

% =============================================================================
% CONFIGURACIÓN DE FORMATO DE TÍTULOS
% =============================================================================
% Personalización visual de secciones, subsecciones y subsubsecciones

% Formato para secciones principales
\titleformat{\section}
    {\color{primaryColor}\Large\bfseries}                    % Estilo del texto
    {\fcolorbox{primaryColor}{primaryColor}{\color{headerTextColor}\thesection}} % Número de sección
    {1em}                                                     % Separación
    {}                                                        % Código antes del título
    [{\titlerule[0.5ex]}]                                    % Línea debajo del título

% Formato para subsecciones
\titleformat{\subsection}
    {\color{primaryColor}\large}                             % Estilo del texto
    {\bfseries\color{primaryColor}\thesubsection}            % Número de subsección
    {1em}                                                     % Separación
    {}                                                        % Código antes del título

% Formato para subsubsecciones
\titleformat{\subsubsection}
    {\color{primaryColor}}                                   % Estilo del texto
    {\bfseries\color{primaryColor}\thesubsubsection}         % Número de subsubsección
    {1em}                                                     % Separación
    {}                                                        % Código antes del título

% =============================================================================
% CONFIGURACIÓN DE ENCABEZADOS Y PIES DE PÁGINA
% =============================================================================
% Sistema personalizado de headers y footers con elementos gráficos

% Eliminar líneas por defecto de fancyhdr
\renewcommand{\headrulewidth}{0pt}
\renewcommand{\footrulewidth}{0pt}

% Comando para crear caja decorativa en el encabezado
\newcommand{\createHeaderBox}{
    \begin{tikzpicture}[overlay, remember picture, x=1in, y=1in, shift=(current page.north west)]
        \fill[primaryColor] (1,-0.25) rectangle (\headerBoxWidth,-0.5);
        \draw[secondaryColor, ultra thick] (0.25,-0.5) -- (6,-0.5);
    \end{tikzpicture}
}

% Comando para el logo en el pie de página
\newcommand{\createFooterLogo}{
    \color{secondaryColor}\textbf{PUCMM}
}

% Configuración de headers y footers
\fancyhead{}                    % Limpiar encabezados
\fancyfoot{}                    % Limpiar pies de página
\fancyfoot[R]{\createFooterLogo}                              % Logo en pie derecho
\fancyhead[L]{\createHeaderBox\color{headerTextColor}\documentShortTitle} % Título en encabezado izquierdo
\fancyhead[R]{\color{bodyTextColor}Pág.\ \thepage\ -~\pageref{LastPage}} % Paginación en encabezado derecho

% =============================================================================
% CONFIGURACIÓN DE TABLA DE CONTENIDOS
% =============================================================================
% Personalización del índice con puntillismo y espaciado profesional

\renewcommand{\cftsecleader}{\cftdotfill{\cftdotsep}}      % Puntillismo para secciones
\renewcommand{\cftsubsecleader}{\cftdotfill{\cftdotsep}}   % Puntillismo para subsecciones
\setlength{\cftbeforesecskip}{0.5em}                       % Espacio antes de secciones
\setlength{\cftbeforesubsecskip}{0.2em}                    % Espacio antes de subsecciones

% =============================================================================
% CONFIGURACIÓN DEL DOCUMENTO - PERSONALIZAR AQUÍ
% =============================================================================
% Variables principales del documento. Modifica estos valores para personalizar
% tu trabajo académico.

% Información institucional
\newcommand{\universityName}{Pontificia Universidad Católica Madre y Maestra}
\newcommand{\departmentName}{[NOMBRE DEL DEPARTAMENTO]}     % Ej: Escuela de Ingeniería
\newcommand{\facultyName}{[NOMBRE DE LA FACULTAD]}         % Ej: Facultad de Ciencias e Ingeniería

% Información del documento
\newcommand{\documentMainTitle}{[TÍTULO PRINCIPAL DEL DOCUMENTO]}     % Título completo
\newcommand{\documentShortTitle}{[TÍTULO CORTO]}                      % Título para headers (máx 20 caracteres)

% Información del curso
\newcommand{\courseName}{[NOMBRE DEL CURSO]}               % Ej: Metodología de la Investigación
\newcommand{\courseCode}{[CÓDIGO DEL CURSO]}               % Ej: CSTI-1890-4341

% Información del estudiante (mantener datos reales)
\newcommand{\studentName}{Cristian de la Hoz}
\newcommand{\studentId}{(0000-0000)}

% Información del instructor
\newcommand{\instructorName}{[NOMBRE DEL PROFESOR]}        % Ej: Prof. Juan Pérez

% Fechas
\newcommand{\submissionDate}{[MES DE AÑO]}                 % Ej: Enero de 2024
\newcommand{\shortDate}{[DD/MM/AAAA]}                      % Ej: 15/01/2024

% Información del logo institucional
\newcommand{\institutionalLogo}{assets/Logo PUCMM (Color).png}    % Ruta al logo de la institución

% =============================================================================
% GENERACIÓN AUTOMÁTICA DE ARCHIVO BIBLIOGRÁFICO
% =============================================================================
% Este bloque crea automáticamente un archivo biblio.bib con entradas de ejemplo
% DESPUÉS DE LA PRIMERA COMPILACIÓN, puedes comentar este bloque si ya tienes
% tu propio archivo de bibliografía.

\begin{filecontents*}{biblio.bib}
% Archivo de bibliografía generado automáticamente
% Reemplaza estas entradas con tus referencias reales

@book{ejemplo2024,
  author = {Apellido, Nombre and Apellido2, Nombre2},
  title = {Título del Libro de Ejemplo},
  year = {2024},
  edition = {1},
  publisher = {Editorial Ejemplo},
  address = {Ciudad, País}
}

@article{ejemploarticulo2024,
  author = {Investigador, Principal},
  title = {Título del Artículo de Investigación},
  journal = {Revista Académica Ejemplo},
  volume = {15},
  number = {3},
  pages = {45-67},
  year = {2024}
}

@online{ejemploweb2024,
  author = {Organización Ejemplo},
  title = {Título del Recurso Web},
  year = {2024},
  url = {https://www.ejemplo.com},
  urldate = {2024-01-15}
}
\end{filecontents*}

% Cargar archivo de bibliografía
\addbibresource{biblio.bib}

% =============================================================================
% CONFIGURACIÓN DE HIPERVÍNCULOS
% =============================================================================
% Personalización de enlaces internos y externos del documento
\hypersetup{
    linktoc=page,                    % Enlaces en números de página del índice
    colorlinks=true,                 % Colorear enlaces en lugar de cajas
    allcolors=linkColor,             % Color uniforme para todos los enlaces
    pdftitle=\documentMainTitle      % Título en metadatos del PDF
}

% =============================================================================
% CÁLCULO AUTOMÁTICO DE DIMENSIONES
% =============================================================================
% Variables calculadas automáticamente para elementos gráficos
\newlength{\headerBoxWidth}
\AtBeginDocument{
    \setlength{\headerBoxWidth}{\widthof{\documentShortTitle}+1.11in}
}

% =============================================================================
% INICIO DEL DOCUMENTO
% =============================================================================
\begin{document}

% =============================================================================
% PÁGINA DE TÍTULO
% =============================================================================
% Portada profesional del documento generada automáticamente
% =============================================================================
% PÁGINA DE TÍTULO PROFESIONAL
% =============================================================================
% Este archivo genera automáticamente una portada profesional usando las
% variables definidas en el documento principal.

\begin{titlepage}
    \thispagestyle{empty} % Sin numeración en la página de título
    \begin{center}
        
        % Logo institucional
        \includegraphics[width=1.5in]{\institutionalLogo}
        \vspace{0.1in}

        % Líneas decorativas debajo del logo
        \noindent\rule{\textwidth}{0.4pt}
        \noindent\rule{\textwidth}{7pt}
        \vspace{0.1in}

        % Nombre de la universidad
        {\Huge\bfseries \universityName}
        \vspace{0.1in}

        % Información del departamento y facultad
        {\Large\bfseries \departmentName}
        
        {\large\bfseries \facultyName}
        \vspace{0.1in}

        % Título principal del documento en caja colorida
        \begin{tcolorbox}[
            sharp corners,
            boxrule=0pt, 
            colframe=secondaryColor, 
            colback=primaryColor, 
            coltext=headerTextColor, 
            width=1.2\textwidth, 
            center, 
            halign=center
        ]
            \Huge\bfseries \documentMainTitle
        \end{tcolorbox}
        \vspace{0.1in}

        % Información del curso
        {\large\bfseries \courseName}
        
        {\bfseries \courseCode}
        \vspace{0.1in}

        \vfill

        % Información de los estudiantes
        {\large\bfseries \studentName}
        
        {\small\bfseries \studentId}

        {\large\bfseries \secondStudentName}

        {\small\bfseries \secondStudentId}

        {\large\bfseries \thirdStudentName}

        {\small\bfseries \thirdStudentId}

        \vfill
        \vspace{0.1in}

        % Información del instructor
        {\large\bfseries Prof. \instructorName}
        \vspace{0.1in}

        % Fecha de entrega
        \bfseries \submissionDate
        
    \end{center}
\end{titlepage}


% =============================================================================
% TABLA DE CONTENIDOS
% =============================================================================
% Índice automático del documento
\newpage
\pagenumbering{Roman}               % Numeración romana para páginas preliminares
\tableofcontents
\newpage

% =============================================================================
% CONTENIDO PRINCIPAL DEL DOCUMENTO
% =============================================================================
% Aquí comienza el contenido académico principal
\pagenumbering{arabic}              % Numeración arábiga para contenido principal
\pagestyle{fancy}                   % Activar encabezados y pies de página personalizados

% -----------------------------------------------------------------------------
% SECCIÓN: INTRODUCCIÓN
% -----------------------------------------------------------------------------
% Presenta el propósito y contexto del trabajo académico
\section{Introducción}

 [ESCRIBIR AQUÍ LA INTRODUCCIÓN DEL DOCUMENTO]

% Ejemplo de introducción:
% Este documento presenta un análisis detallado sobre [TEMA PRINCIPAL]. 
% El objetivo es [OBJETIVO PRINCIPAL] mediante [METODOLOGÍA UTILIZADA].
% La estructura del trabajo incluye [DESCRIPCIÓN DE SECCIONES].

% -----------------------------------------------------------------------------
% SECCIÓN: DESARROLLO PRINCIPAL
% -----------------------------------------------------------------------------
% Contenido principal del trabajo académico
\section{Desarrollo}

% Subsección de ejemplo
\subsection{[Título de Subsección]}

[CONTENIDO DE LA SUBSECCIÓN]

% Ejemplo de cómo citar fuentes
% Según \cite{ejemplo2024}, el tema de investigación presenta características importantes...

% Otra subsección de ejemplo
\subsection{[Otra Subsección]}

[CONTENIDO DE OTRA SUBSECCIÓN]

% Subsubsección de ejemplo
\subsubsection{[Tema Específico]}

[CONTENIDO ESPECÍFICO]

% -----------------------------------------------------------------------------
% SECCIÓN: ANÁLISIS Y RESULTADOS (OPCIONAL)
% -----------------------------------------------------------------------------
% Incluir esta sección para trabajos de investigación o análisis
\section{Análisis y Resultados}

 [PRESENTAR AQUÍ EL ANÁLISIS DE LOS DATOS O RESULTADOS OBTENIDOS]

% -----------------------------------------------------------------------------
% SECCIÓN: DISCUSIÓN (OPCIONAL)
% -----------------------------------------------------------------------------
% Interpretación de resultados y conexión con literatura existente
\section{Discusión}

 [INTERPRETAR Y DISCUTIR LOS RESULTADOS EN RELACIÓN CON EL CONTEXTO ACADÉMICO]

% -----------------------------------------------------------------------------
% SECCIÓN: CONCLUSIONES
% -----------------------------------------------------------------------------
% Síntesis final del trabajo y recomendaciones
\section{Conclusiones}

 [ESCRIBIR AQUÍ LAS CONCLUSIONES PRINCIPALES DEL TRABAJO]

% Ejemplo de estructura para conclusiones:
% 1. Resumen de hallazgos principales
% 2. Implicaciones teóricas o prácticas
% 3. Limitaciones del estudio
% 4. Recomendaciones para trabajo futuro

% =============================================================================
% BIBLIOGRAFÍA
% =============================================================================
% Referencias bibliográficas automáticas
\newpage
\pagestyle{empty}                   % Sin encabezados ni pies de página
\printbibliography[heading=bibintoc, title={Referencias}]

\end{document}
